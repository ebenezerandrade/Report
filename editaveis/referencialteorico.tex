\chapter[O Tribunal de Contas da União]{O Tribunal de Contas da União}

\section{História}
O TCU é um orgão adminstrativo autônomo e independetente que auxilia o Congresso Nacional (CN) em realizar o controle externo dos recursos públicos federais \cite{TCUHistoria}. Além disso o orgão possui sede em Brasília no Distrito Federal (DF) e jurisdição em todo território nacional \cite{Art73}. 

A instituição teve seu surgimento pela Constituição de \textbf{1891}, sob influência de Rui Barbosa, institucionalizou definitivamente o TCU, mas sua instalação só ocorreu em 17 de janeiro de \textbf{1893}, graças ao emprenho do Ministro da Fazenda do governo de Floriano Peixoto, Serzedello Corrêa\cite{TCUHistoriaJUS}. Logo, o TCU teve a competência para exame, revisão e julgamento de todas as operações relacionadas com a receita e a despesa da União. Nos anos seguintes as responsabilidades do TCU foram acrescentadas\cite{TCUHistoriaJUS}:

\begin{itemize}
	\item \textbf{1937:} Com exceção do parecer sobre as contas da presidência, todas as atribuições ao tribunal foram mantidas;
	
	\item \textbf{1946:} Acrescentou o julgamento a legalidade das concessões de aponsentadorias, reformas e pensões;
	
	\item \textbf{1967:} De acordo com a Constituição de \textbf{1967}, retirou do TCU o exame e julgamento prévio dos atos e dos contratos geradores de despesas, sem prejuízo da competência para apontar falhas e irregularidades; 
\end{itemize}

Então de acordo com Luiz Eduardo Oliveira Alejarra, no ano de \textbf{1988} o TCU passou a obter: 

\begin{citacao}
	o Tribunal de Contas da União teve a sua jurisdição e competência substancialmente ampliadas. Recebeu poderes para, no auxílio ao Congresso Nacional, exercer a fiscalização contábil, financeira, orçamentária, operacional e patrimonial da União e das entidades da administração direta e indireta, quanto à legalidade, à legitimidade e à economicidade e a fiscalização da aplicação das subvenções e da renúncia de receitas. Qualquer pessoa física ou jurídica, pública ou privada, que utilize, arrecade, guarde, gerencie ou administre dinheiros, bens e valores públicos ou pelos quais a União responda, ou que, em nome desta, assuma obrigações de natureza pecuniária tem o dever de prestar contas ao TCU \cite[p. 1]{TCUHistoriaJUS}.
\end{citacao}

\pagebreak

\section{Logo}

A logo atual do Tribunal de Contas é apresentada pela figura \ref{logoTCU}.  

\begin{figure}[h!]
	\centering
	\includegraphics[keepaspectratio=true,scale=0.5]{figuras/TCU.jpg}
	\caption{Logo do TCU. Fonte: \cite{TCUHistoria}}
	\label{logoTCU}
\end{figure}

\section{Endereço}

O endereço da sede do TCU onde o estágio exerceu as atividades:
St. de Administração Federal Sul, Quadra:4 Lote:01 - Asa Sul, Brasília - DF, 70042-900 \cite{TCUHistoria}



\chapter[Atividades Desenvolvidas e Cronograma de Execução]{Atividades Desenvolvidas e Cronograma de Execução}

Neste Cápitulo é apresentado as atividades desenvolvidas de acordo com o Plano de Atividades, anexado ao Termo Aditivo do Contrato. A descrição das atividades que foram propostas a serem desenvolvidas e a carga horária semanal é apresentada na tabela \ref{AtividadesDaPA} 

\begin{table}[h]
	\centering
	\caption{Descrição das atividades propostas no Plano de Atividades}
	\label{AtividadesDaPA}
	\begin{tabular}{|c|c|c|}
		\hline
		\textbf{Identificador} & \textbf{Atividade} & \textbf{\begin{tabular}[c]{@{}c@{}}Carga horária\\ Semanal\end{tabular}} \\ \hline
		1 & \begin{tabular}[c]{@{}c@{}}Auxiliar no desenvolvimento de sistemas em\\ Oracle Application Express, abrangendo as \\ atividades de levantamento de requisitos,\\ modelagem de banco de dados, implementação,\\ teste, documentação e manutenção\end{tabular} & 4h \\ \hline
		2 & Auxiliar a manutenção de sistemas (Sidarq) & 10h \\ \hline
		3 & Auxiliar organização de sistemas & 5h \\ \hline
		4 & Auxiliar as áreas de suporte & 1h \\ \hline
	\end{tabular}
\end{table}

\chapter[Resultados e Discussão]{Resultados e Discussão}



\chapter[Considerações Finais]{Considerações Finais}