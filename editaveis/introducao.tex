\chapter{Introdução}

Atualmente as organizações que realizam em seus processos internos desenvolvimento e manutenção de software, atuam com a preocupação de como melhorar esse produto de software de acordo com as reais necessidades. Logo o compreendimento das práticas da fundamentais da Manutenção e Evolução de Software, auxiliam a melhores esforços quando precisam manter seus produtos \cite{SQLMagazine}.

Entretando  segundo \cite{pfleeger2004engenharia} as atividades da Manutenção e Evolução de Software são semelhantes a atividades do desenvolvimento como(i) Analisar os requisitos, (ii) avaliar o sistema e o projeto, (iii)  programar e revisar o código, (iv) testar as modificações e (v) atualizar a documentação.
 
A Manutenção de Software por ser divida em 4 tipos \cite{pfleeger2004engenharia}:

\begin{itemize}
	
	\item \textbf{Manutenção corretiva:} Conforme vão ocorrendo as falhas elas são relatadas à equipe de manutenção, que encontra a causa da falha e realiza as correções necessárias nos (i) requisitos, (ii) no projeto, (iii) no código, (iv) no conjunto de testes e (v) na documentaçao.
	
	\item \textbf{Manutenção adaptativa:}  Tipo de manutenção quando uma mudança é introduzida em uma parte do sistema e requer a modificação em outras partes, esse tipo de mudança também pode ser realizada no hardware ou no ambiente.
	
	\item \textbf{Manutenção perfectiva:} Conforme vão ocorrendo as mudanças no sistema, examina-se os (i) documentos, (ii) o código e (iii) os testes. Esse tipo de manutenção consiste em realizar mudanças para melhorar alguns aspectos do sistema, mesmo quando nenhuma das mudanças for devido ao surgimento de algum defeito.
	
	\item \textbf{Manutenção preventiva} Consiste em modificar alguns aspectos do sistema, com intuito de prevenir falhas. 
	
\end{itemize}

Logo este documento apresenta o relatório técnico detalhando as atividades realizadas na Central de Documentos (CDOC), do Tribunal de Contas da União (TCU), Para realização das atividades de manutenção e evolução do Sistema de Arquivamento de Processos (Sidarq).

\section{Objetivos}

\subsection{Objetivo Geral}

Apresentar as atividades desenvolvidas pelo estagiário durante o período de estágio supervisionado, acordadas no contrato vigente entre o período de 26/06/2017 a 25/12/2017. As atividades foram supervisionadas no TCU por Paulo André Mattos de Carvalho e Ana Claudia de Carvalho Cabral Lopes.

\subsection{Objetivos Específicos}

\begin{itemize}
	\item Detalhar procedimentos realizados durante a execução das atividades no TCU.
\end{itemize}
